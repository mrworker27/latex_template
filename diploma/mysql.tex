\section{Поддержка диалекта MySQL}
\subsection{Парсер MySQL}
Про то, какой был выбран парсер, что такое ANTLR, как он компилируется, еще раз про БНФ немного, про то, что переписал с C++, про рантайм пару слов
\subsection{MySQL AST}
Я ввожу новый уровень абстракции, предназначеный для изолятии синтаксического дерева, полученного из ANTLR. Вводится новая легкая сущность, повторяющая структуру исходного дерева, но содерджащая только необходимую информацию
\subsection{Дерево конвертации}
Тут про то, как я из одного дерева перекукоживаю запрос в другое. Новая древовидная структура с необходимыми свойствами. Налить водищи
