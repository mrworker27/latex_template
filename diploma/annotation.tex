\specialsection{Аннотация}
Множество систем управления базами данных (СУБД) поддерживают свой диалект SQL (язык, описывающий манипуляции с данными), при помощи которого выстраивается интерфейс взаимодействия с внешним клиентом. Диалекты SQL разных СУБД имеют различную совместимость между собой и стандартом SQL, что в некоторых случаях приводит к тому, что внешние инструменты, успешно работающие с одной СУБД, не могут работать с другой только из-за несовместимости диалектов SQL. Данная работа ставит перед собой цель разработать метод поддержки стороннего диалекта SQL в СУБД на примере поддержки диалекта СУБД MySQL в СУБД ClickHouse. В работе предложена концепция деревьев конвертации, описывающая метод поддеркжи стороннего диалекта путем конвертации абстрактных синтаксических деревьев. Данный метод был применен в ClickHouse, результатом чего стала его совместимость с подмножеством запросов SELECT на диалекте SQL в ClickHouse.

\specialsection{Abstract}

A lot of database management systems use thier own dialect of SQL (language, that expresses manipulations with data) to provide an interaction interface with external client. Different SQL dialect can have differenct level of compatibility with each other and with SQL standard, wich leads to the situation, where one external instrument can successfully interact with one DBMS, but fails to interract with another only because of SQL-dialect incompatibility. The goal of this paper is to develop exteranal SQL-dialect support method which will be applied to support DBMS MySQL dialect within DBMS ClickHouse. This paper proposes the concept of conversion tree, that can be used to support external SQL-dialect by doing abstract syntax tree conversion. This method was implemented into ClickHouse, which was resulted in partial support of SELECT queires in MySQL dialect inside ClickHouse

\textbf{Ключевые слова}: ClickHouse, MySQL, парсер, дерево разбора, AST, абстрактное синтаксическое дерево, дерево конвертации, SQL, диалект SQL

\textbf{Keywords}: ClickHouse, MySQL, parser, parse tree, AST, abstract syntax tree, conversion tree, SQL, SQL dialect
