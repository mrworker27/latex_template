\section{Введение}
\subsection{Актуальность}
Существует множество способов взаимодействия между системой управления базами данных (СУБД) и внешним клиентом (пользователь или другая программа). Наиболее распространенным из них является взаимодействие через специальный язык запросов - SQL. Не смотря на то, что SQL является самодостаточным стандартизированным языком, на практике большинство СУБД поддерживают свой диалект SQL. Каждый конкретный диалект в разной степени совместим с SQL: он может как поддерживать дополнительный функционал, не предусмотренный SQL, так и не обладать поддержкой некоторых возможностей SQL. Поддержка собственного диалекта SQL обусловлена необходимостью выразить особенности СУБД на уровне интерфейса взаимодействия.

Наличие множества диалектов SQL напрямую ставит вопрос о совместимости различных СУБД на уровне интерфейса взаимодействия со внешним клиентом. В частности, различия диалектов приводят к тому, что затрудняют разработку решений, нацеленных на взаимодействие не с конкретной СУБД, а с СУБД в общем виде (более точно - с некоторым классом СУБД). Одним из возможных решений обозначенной проблемы является внедрение в СУБД поддержки сторонних диалектов SQL наряду с собственным. Именно этой теме посвящена данная работа

\subsection{Краткое описание работы}
Работа будет посвящена поддержке SQL диалекта СУБД MySQL в аналитической СУБД ClickHouse. ClickHouse, как и многие другие СУБД, поддерживает собственный диалект SQL во многом совместимый как с SQL в целом, так и с MySQL диалектом SQL в частности. Однако, существует множество специфичных различий, которые ведут к несовместимости ClickHouse и MySQL в ряде случаев. Поскольку MySQL является наиболее распространенной СУБД на момент написания работы, существенное множество сторонних инструментов ориентированных на взаимодействие с MySQL. Поддержка MySQL диалекта SQL в ClickHouse позволит ClickHouse успешно взаимодействовать с этими инструментами, что увеличит его возможную область применения.  

\subsection{Цели и задачи}
Целью данной работы является внедрение парсера диалекта MySQL в ClickHouse и поддержка запросов \mintinline{sql}{ SELECT } на диалекте MySQL. 

Для достижения цели были поставлены следующие задачи:
\begin{itemize}
  \item Внедрить в кодовую базу ClickHouse парсер диалекта MySQL с подходящей лицензией
  \item Добавить юнит-тесты, проверяющие корректность работы парсера диалекта MySQL внутри ClickHouse
  \item Разработать метод преобразования запросов на диалекте MySQL в диалект ClickHouse, использующий конвертацию абстрактных синтаксических деревьев
  \item С помощью разработанного метода реализовать поддержку запрсов \mintinline{sql}{ SELECT } на диалекте MySQL в ClickHouse
  \item Добавить функциональные тесты, проверяющие корректность преобразования запросов на диалекте MySQL в ClickHouse
  \item Исследовать совместимость ClickHouse, поддерживающего диалект MySQL, со сторонними инструментами, ориентированным на работу с MySQL
\end{itemize}

\subsection{Полученные результаты}
Результами дипломной работы стали:

\begin{itemize}
    \item Успешное внедрение парсера MySQL в кодовую базу ClickHouse
    \item Метод конвертации абстрактных синтаксических деревьев, позволяющий итеративно увеличивать совместимость ClickHouse с диалектом MySQL
    \item Поддержка ClickHouse широкого подмножества запросов \mintinline{sql}{ SELECT } на диалекте MySQL
    \item Совместимость ClickHouse, работающего в режиме распознавания диалекта MySQL, с Google Data Studio (веб-сервис, позволяющий визуализировать данные под)
\end{itemize}

Более подробно полученные результаты описаны в соответсвующей главе (см. Главу \ref{chap:results})

\subsection{Структура работы}
Данная работа имеет следущую структуру:
\begin{itemize}
    \item В главе 2 изложен обзор предметной области: краткий обзор используемых программ и инструментов (ClickHouse, MySQL, ANTLR), описание понятий и методов, необходмых для понимания теоретической основы работы (формальные языки, форма Бэкуса-Наура, лексеры и парсеры, синтаксическое дерево, AST, SQL).
    \item В главе 3 изложен краткий обзор архитектуры ClickHouse, отвечающей за разбор запроса
    \item В главе 4 изложена структура выбранного парсера MySQL и описан процесс его интеграции в кодовую базу ClickHouse
    \item В главе 5 предложена концепция \textit{деревьев конвертации}, а так же изложена архитектура конвертации абстрактных синтаксических деревьев
    \item В главе 6 описаны методы тестирования полученного решения
    \item В главе 7 описаны полученные результаты и дальнейшие направления работы
\end{itemize}
