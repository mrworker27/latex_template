\section{Введение}
\subsection{Актуальность}
Существует множество способов взаимодействия между системой управления базами данных (СУБД) и внешним клиентом (пользователь или другая программа). Наиболее распространенным из них является взаимодействие через специальный язык запросов - SQL. Не смотря на то, что SQL является самодостаточным стандартизированным языком, на практике большинство СУБД поддерживают свой диалект SQL. Каждый конкретный диалект в разной степени совместим с SQL: он может как поддерживать дополнительный функционал, не предусмотренный SQL, так и не обладать поддержкой некоторых возможностей SQL. Поддержка собственного диалекта SQL обусловлена необходимостью выразить особенности СУБД на уровне интерфейса взаимодействия.

Наличие множества диалектов SQL напрямую ставит вопрос о совместимости различных СУБД на уровне интерфейса взаимодействия со внешним клиентом. В частности, различия диалектов приводят к тому, что затрудняют разработку решений, нацеленных на взаимодействие не с конкретной СУБД, а с СУБД в общем виде (более точно - с некоторым классом СУБД). Одним из возможных решений обозначенной проблемы является внедрение в СУБД поддержки сторонних диалектов SQL наряду с собственным. Именно этой теме посвящена данная работа

\subsection{Краткое описание работы}
Работа будет посвящена поддержке SQL диалекта СУБД MySQL в аналитической СУБД ClickHouse. ClickHouse, как и многие другие СУБД, поддерживает собственный диалект SQL во многом совместимый как с SQL в целом, так и с MySQL диалектом SQL в частности. Однако, существует множество специфичных различий, которые ведут к несовместимости ClickHouse и MySQL в ряде случаев. Поскольку MySQL является наиболее распространенной СУБД на момент написания работы, существенное множество сторонних инструментов ориентировано на взаимодействие с MySQL. Поддержка MySQL диалекта SQL в ClickHouse позволит ClickHouse успешно взаимодействовать с этими инструментами, что увеличит его возможную область применения.  

\subsection{Цели и задачи}
Целью данной работы является внедрение парсера диалекта MySQL в ClickHouse и поддержка запросов \lstinline[style=customsql]|SELECT| на диалекте MySQL. 

Для достижения цели были поставлены следующие задачи:
\begin{itemize}
  \item Внедрить в кодовую базу ClickHouse парсер диалекта MySQL с подходящей лицензией
  \item Добавить юнит-тесты, проверяющие корректность работы парсера диалекта MySQL внутри ClickHouse
  \item Разработать метод преобразования запросов на диалекте MySQL в диалект ClickHouse, использующий конвертацию абстрактных синтаксических деревьев
  \item С помощью разработанного метода реализовать поддержку запрсов \lstinline[style=customsql]|SELECT| на диалекте MySQL в ClickHouse
  \item Добавить функциональные тесты, проверяющие корректность преобразования запросов на диалекте MySQL в ClickHouse
  \item Исследовать совместимость ClickHouse, поддерживающего диалект MySQL, со сторонними инструментами, ориентированным на работу с MySQL
\end{itemize}

\subsection{Полученные результаты}
ЭТО НАПИШУ В КОНЦЕ НА ОСНОВАНИИ САМОЙ РАБОТЫ

\subsection{Структура работы}
ЭТО НАВЕРНОЕ БУДУ ЗАПОЛНЯТЬ И ИЗМЕНЯТЬ ПРЯМ ПО ХОДУ САМОЙ РАБОТЫ
