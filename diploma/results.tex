\section{Результаты} \label{chap:results}
Непосредственным результатом дипломной разботы является Pull Request в основной репозиторий ClickHouse, содержащий код, интегрирующий парсер MySQL в ClickHouse и позволяющий ClickHouse работать в режиме диалекта MySQL, распознавая некоторое подмножество запросов на этом диалекте

\subsection{Метод конвертации}
В рамках данной работы не удалось поддержать в ClickHouse синтаксис MySQL целиком, более того, большинство типов запросов не распознаются на начальном этапе с последующим возвратом ошибки (см. Раздел \ref{conv:recognizer_}).

Несмотря на это, в работе предложена концепция \textit{деревьев конвертации}, которая была реализована в обозначенном выше Pull Request в качестве метода конвертации AST. Ее использование позволит постепенно увеличивать совместимсть ClickHouse с диалектом MySQL вплоть до полной совместимости (возможно, ограниченной факторами, не относящимися непосредственно к языку запросов). 

\subsection{Совместимость с Google Data Studio} \label{res:google}
Корректность представленного в данной работе решения может быть проверена не только синтетическими авто-тестами, которые можно найти в Pull Request. Совместимость ClickHouse с диалектом MySQL была проверена путем интеграции ClickHouse c Google Data Studio (веб-сервис, позволяющий визуализировать данные под управлением СУБД)

На удаленном хосте, доступном из внешней сети, была размещена программа \textit{clickhouse}, скомпилированная из ветки, содержащей изменения, описанные в данной работе (описанные в Pull Request, упомянутом выше). Был запущен сервер ClickHouse с конфигурационным файлом, предписывающим ClickHouse использовать диалект MySQL по умолчанию для всех подключений (см. Раздел \ref{conv:general_}). С использованием собственного диалекта, в ClickHouse были созданы тестовые таблицы вместе с содержимым

После этого Google Data Studio была подключена к серверу ClickHouse посредством \enquote{коннектора MySQL}. Затем была создана страница, включающие в себя элементы визуализации, которые предоставляет Google Data Studio (TODO: ССЫЛКА КОТОРАЯ В ТЕЛЕГЕ). ClickHouse в режиме работы с MySQL диалектом успешно обработал запросы, полученные от Google Data Studio, что говорит корректности реализованного решения

В ходе интеграции ClickHouse в режиме работы с диалектом MySQL обнаружилось, что Google Data Studio не может успешно подключться к серверу через графический интрфейс, хотя подключение через запрос происходит успешно. Применяя \textit{reverse engineering}, было установлено, что проблема заключается в том, что Google Data Studio не может успешно обработать ответ на запрос вида \mintinline{sql}{ DECRIBE table }, хотя ClickHouse его обрабатывает без ошибок и возвращает корректный результат. Прична такого поведения - несоответсвие \textit{формата вывода} ответа на данный запрос, что выходит за рамки данной работы. 

\subsection{Дальнейшее направление работы}
Данная работа предлагает метод конвертации AST и демонстрирует его работу на примере реализации поддержки диалекта MySQL в ClickHouse, что делает ее завершенной и самодостаточной

Однако, за рамками дипломной работы остается множество направлений для дальнейшей разработки, которые станут частью работы по внедрению полученных результатов в основной репозиторий ClickHouse:
\begin{enumerate}
    \item Поддержать новые типы запросов (\mintinline{sql}{ UPDATE }, \mintinline{sql}{ INSERT }, \mintinline{sql}{ DELETE } и т. д.)
    \item Улучшить совместимость ClickHouse с уже поддерживаемым типами запросов (например добавить поддержку \mintinline{sql}{ JOIN } и \mintinline{sql}{ UNION } в запросы \mintinline{sql}{ SELECT }
    \item Поддержать альтернативные форматы вывода в зависимости от выбранного диалекта SQL (с целью решить проблему, обозначенную в конце Раздела \ref{res:google})
    \item Расширить список сторонних инструментов, совместимых с ClickHouse, работающем в режиме поддержки диалекта MySQL (phpMyAdmin, adminer, Sequel Ace и т. д.)
\end{enumerate}

\pagebreak
