\section{Поддержка диалекта MySQL в ClickHouse}
\subsection{Общая идея}
Про то, куда вонзаемся, схематично. Смотрим как делает парсер, смотрим на наше дерево и кукожим

\subsection{Распознание запроса}
Нам пришло непонятное дерево, у ClickHouse AST не едины типом своего корня. Первым шагом нужно распознать, какой тип запроса пришел, а дальше уже отдать его в соответсвующее правило

\subsection{Деревья конвертации}
Вводим новую сущность - дерево конвертации, две метода - сетап и конверт. Каждая вершина конвертации порождает ровно одну вершину ClickHouse AST (возможно, с потомками). Сетпап может завершиться с ошибкой, convert - нет. Сетап извлекает данные из MySQL AST и проверяет его структуру. И т д и т п

\subsection{Множество поддерживаемых запросов}
Тут прям по пунктикам выписываю какие запросы поддерживаются. 

\pagebreak
