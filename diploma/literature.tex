\section{Обзор предметной области и существующих решений}
\subsection{ClickHouse}
TODO: написать про то, что такое ClickHouse и зачем он нужен.

\subsection{MySQL}
TODO: написать про то, что такое MySQL и зачем он нужен

\subsection{Формальные языки вцелом}
TODO: дать определение фомральноо языка, контекстно-свободного формального языка

\subsection{SQL}
TODO: рассказать про суть SQL, уточнить, что это контекстно-свободный формальный язык

\subsection{Лексический анализ}
TODO: дать определение лексическому анализу, лексеме, токену

\subsection{Синтаксический анализ}
TODO: дать определение синтсаксическому анализу, форме Бэкуса-Наура (+ расширенная), синтаксическому дереву

\subsection{Абстрактное синтаксическое дерево}
TODO: ввести понятие абстрактного синтаксического дерева, аббревиатуру AST, сказать, что данная работа будет по большей части оперировать именно с AST

\subsection{ANTLR}
TODO: для поддержки MySQL диалекта в ClickHouse будет внедрен парсер MySQL использующий ANTLR для автоматической генерации исходного кода анализаторов. Объяснить что такое ANTLR и как устроена генерация (вкраце)

\subsection{Обзор существующих решений}
TODO: я не успею предметно разобраться в существующих решениях, поэтому рассказ о них я не стал выделять в отдельную главу. Отмечу, что практика поддержки стронних SQL диалектов встречается не так часто, но постепенно обретает все большее распространение. Привету примеры СУБД, в которых эта поддержка реализована (список)

\subsection{Список терминов}
Дать определения прочим терминам, не являющимся ключевыми в данной работе, но которые знать для того, чтобы понять работу
\begin{enumerate}
    \item Юнит-тесты - ...
    \item Функциональные тесты - ...
    \item Скрипт - ...
    \item Вычитать работу и добавить сюда термины
\end{enumerate}

TODO: расставить здесь и в дальнейшем по работе ссылки на источники (пока что они не добавлены)

\pagebreak
